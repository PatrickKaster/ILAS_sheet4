	% HMC Math dept HW template example
	% v0.04 by Eric J. Malm, 10 Mar 2005
	\documentclass[10pt,a4paper,boxed]{hmcpset}

	% set 1-inch margins in the document
	% \usepackage[margin=1in]{geometry}
	\usepackage{enumerate}
	\usepackage{tikz}
	\usetikzlibrary{positioning}
	\usepackage{pgfplots}
	\usepackage{amsmath}
	\usepackage{amsfonts}
	\usepackage{amssymb}

	% include this if you want to import graphics files with /includegraphics
	\usepackage{graphicx}

	\renewcommand*{\familydefault}{\sfdefault}
	\newcommand{\vect}[1]{\mathbf{#1}}
	\DeclareMathOperator{\gain}{Gain}
	\DeclareMathOperator{\entropy}{H}
	\DeclareMathOperator{\prob}{P}

	\tikzset{node distance=2cm, inner/.style={draw,circle}, leaf/.style={draw,rectangle}}

	% info for header block in upper right hand corner
	\name{Group 6: Timm Behner, Philipp Bruckschen, Patrick Kaster, Markus Schwalb}
	\class{MA-INF 4111 - Intelligent Learning and Analysis Systems: Machine Learning}
	\assignment{Exercise Sheet 4}
	% \duedate{09/03/2004}

	\begin{document}

		\begin{problem}[4. Distances]
		\end{problem}
		\begin{solution}
			done. will write up in the evening
	
			For $\delta(x,y)$ to be a metric, we have to show:
			\begin{enumerate}
				\item $\delta(x,y)\geq 0, \delta(x,y)=0 \mbox{ iff } x=y$
				\item $\delta(x,y)=\delta(y,x)$
				\item $\delta(x,y) \leq \delta(x,z)+\delta(z,y)$
			\end{enumerate}
	
			\begin{enumerate}[(i)]
				\item done
				\item dfdf		
			\end{enumerate}
		\end{solution}

	\end{document}